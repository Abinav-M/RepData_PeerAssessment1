\PassOptionsToPackage{unicode=true}{hyperref} % options for packages loaded elsewhere
\PassOptionsToPackage{hyphens}{url}
%
\documentclass[]{article}
\usepackage{lmodern}
\usepackage{amssymb,amsmath}
\usepackage{ifxetex,ifluatex}
\usepackage{fixltx2e} % provides \textsubscript
\ifnum 0\ifxetex 1\fi\ifluatex 1\fi=0 % if pdftex
  \usepackage[T1]{fontenc}
  \usepackage[utf8]{inputenc}
  \usepackage{textcomp} % provides euro and other symbols
\else % if luatex or xelatex
  \usepackage{unicode-math}
  \defaultfontfeatures{Ligatures=TeX,Scale=MatchLowercase}
\fi
% use upquote if available, for straight quotes in verbatim environments
\IfFileExists{upquote.sty}{\usepackage{upquote}}{}
% use microtype if available
\IfFileExists{microtype.sty}{%
\usepackage[]{microtype}
\UseMicrotypeSet[protrusion]{basicmath} % disable protrusion for tt fonts
}{}
\IfFileExists{parskip.sty}{%
\usepackage{parskip}
}{% else
\setlength{\parindent}{0pt}
\setlength{\parskip}{6pt plus 2pt minus 1pt}
}
\usepackage{hyperref}
\hypersetup{
            pdftitle={Reproducible Research: Peer Assessment 1},
            pdfborder={0 0 0},
            breaklinks=true}
\urlstyle{same}  % don't use monospace font for urls
\usepackage[margin=1in]{geometry}
\usepackage{graphicx,grffile}
\makeatletter
\def\maxwidth{\ifdim\Gin@nat@width>\linewidth\linewidth\else\Gin@nat@width\fi}
\def\maxheight{\ifdim\Gin@nat@height>\textheight\textheight\else\Gin@nat@height\fi}
\makeatother
% Scale images if necessary, so that they will not overflow the page
% margins by default, and it is still possible to overwrite the defaults
% using explicit options in \includegraphics[width, height, ...]{}
\setkeys{Gin}{width=\maxwidth,height=\maxheight,keepaspectratio}
\setlength{\emergencystretch}{3em}  % prevent overfull lines
\providecommand{\tightlist}{%
  \setlength{\itemsep}{0pt}\setlength{\parskip}{0pt}}
\setcounter{secnumdepth}{0}
% Redefines (sub)paragraphs to behave more like sections
\ifx\paragraph\undefined\else
\let\oldparagraph\paragraph
\renewcommand{\paragraph}[1]{\oldparagraph{#1}\mbox{}}
\fi
\ifx\subparagraph\undefined\else
\let\oldsubparagraph\subparagraph
\renewcommand{\subparagraph}[1]{\oldsubparagraph{#1}\mbox{}}
\fi

% set default figure placement to htbp
\makeatletter
\def\fps@figure{htbp}
\makeatother


\title{Reproducible Research: Peer Assessment 1}
\author{}
\date{\vspace{-2.5em}}

\begin{document}
\maketitle

\hypertarget{loading-and-preprocessing-the-data}{%
\subsection{Loading and preprocessing the
data}\label{loading-and-preprocessing-the-data}}

library(``data.table'') library(ggplot2)

fileUrl \textless{}-
``\url{https://d396qusza40orc.cloudfront.net/repdata\%2Fdata\%2Factivity.zip}''
download.file(fileUrl, destfile = paste0(getwd(),
`/repdata\%2Fdata\%2Factivity.zip'), method = ``curl'')
unzip(``repdata\%2Fdata\%2Factivity.zip'',exdir = ``data'')

\hypertarget{reading-the-data}{%
\subsection{reading the data}\label{reading-the-data}}

activityDT \textless{}- data.table::fread(input = ``data/activity.csv'')

\hypertarget{what-is-mean-total-number-of-steps-taken-per-day}{%
\subsection{What is mean total number of steps taken per
day?}\label{what-is-mean-total-number-of-steps-taken-per-day}}

Total\_Steps \textless{}- activityDT{[}, c(lapply(.SD, sum, na.rm =
FALSE)), .SDcols = c(``steps''), by = .(date){]}

head(Total\_Steps, 10) ggplot(Total\_Steps, aes(x = steps)) +
geom\_histogram(fill = ``blue'', binwidth = 1000) + labs(title = ``Daily
Steps'', x = ``Steps'', y = ``Frequency'') \#\# Warning: Removed 8 rows
containing non-finite values (stat\_bin). Total\_Steps{[}, .(Mean\_Steps
= mean(steps, na.rm = TRUE), Median\_Steps = median(steps, na.rm =
TRUE)){]}

\hypertarget{what-is-the-average-daily-activity-pattern}{%
\subsection{What is the average daily activity
pattern?}\label{what-is-the-average-daily-activity-pattern}}

IntervalDT \textless{}- activityDT{[}, c(lapply(.SD, mean, na.rm =
TRUE)), .SDcols = c(``steps''), by = .(interval){]}

ggplot(IntervalDT, aes(x = interval , y = steps)) +
geom\_line(color=``blue'', size=1) + labs(title = ``Avg. Daily Steps'',
x = ``Interval'', y = ``Avg. Steps per day'')

IntervalDT{[}steps == max(steps), .(max\_interval = interval){]} \#\#
Imputing missing values activityDT{[}is.na(steps), .N {]}
activityDT{[}is.na(steps), ``steps''{]} \textless{}- activityDT{[},
c(lapply(.SD, median, na.rm = TRUE)), .SDcols = c(``steps''){]}
data.table::fwrite(x = activityDT, file = ``data/tidyData.csv'', quote =
FALSE) \# total number of steps taken per day Total\_Steps \textless{}-
activityDT{[}, c(lapply(.SD, sum)), .SDcols = c(``steps''), by =
.(date){]}

\hypertarget{mean-and-median-total-number-of-steps-taken-per-day}{%
\section{mean and median total number of steps taken per
day}\label{mean-and-median-total-number-of-steps-taken-per-day}}

Total\_Steps{[}, .(Mean\_Steps = mean(steps), Median\_Steps =
median(steps)){]}

\hypertarget{are-there-differences-in-activity-patterns-between-weekdays-and-weekends}{%
\subsection{Are there differences in activity patterns between weekdays
and
weekends?}\label{are-there-differences-in-activity-patterns-between-weekdays-and-weekends}}

\hypertarget{just-recreating-activitydt-from-scratch-then-making-the-new-factor-variable.-no-need-to-just-want-to-be-clear-on-what-the-entire-process-is.}{%
\section{Just recreating activityDT from scratch then making the new
factor variable. (No need to, just want to be clear on what the entire
process
is.)}\label{just-recreating-activitydt-from-scratch-then-making-the-new-factor-variable.-no-need-to-just-want-to-be-clear-on-what-the-entire-process-is.}}

activityDT \textless{}- data.table::fread(input = ``data/activity.csv'')
activityDT{[}, date := as.POSIXct(date, format = ``\%Y-\%m-\%d''){]}
activityDT{[}, \texttt{Day\ of\ Week}:= weekdays(x = date){]}
activityDT{[}grepl(pattern =
``Monday\textbar{}Tuesday\textbar{}Wednesday\textbar{}Thursday\textbar{}Friday'',
x = \texttt{Day\ of\ Week}), ``weekday or weekend''{]} \textless{}-
``weekday'' activityDT{[}grepl(pattern = ``Saturday\textbar{}Sunday'', x
= \texttt{Day\ of\ Week}), ``weekday or weekend''{]} \textless{}-
``weekend'' activityDT{[}, \texttt{weekday\ or\ weekend} :=
as.factor(\texttt{weekday\ or\ weekend}){]} head(activityDT, 10)

activityDT{[}is.na(steps), ``steps''{]} \textless{}- activityDT{[},
c(lapply(.SD, median, na.rm = TRUE)), .SDcols = c(``steps''){]}
IntervalDT \textless{}- activityDT{[}, c(lapply(.SD, mean, na.rm =
TRUE)), .SDcols = c(``steps''), by = .(interval,
\texttt{weekday\ or\ weekend}){]}

ggplot(IntervalDT , aes(x = interval , y = steps,
color=\texttt{weekday\ or\ weekend})) + geom\_line() + labs(title =
``Avg. Daily Steps by Weektype'', x = ``Interval'', y = ``No.~of
Steps'') + facet\_wrap(\textasciitilde{}\texttt{weekday\ or\ weekend} ,
ncol = 1, nrow=2) echo=TRUE

\end{document}
